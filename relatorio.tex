\documentclass[a4paper, 12pt]{article}
\usepackage{lmodern}
\usepackage{amssymb,amsmath}
\usepackage{ifxetex,ifluatex}
\usepackage{fixltx2e} % provides \textsubscript
\ifnum 0\ifxetex 1\fi\ifluatex 1\fi=0 % if pdftex
  \usepackage[T1]{fontenc}
  \usepackage[utf8]{inputenc}
\else % if luatex or xelatex
  \ifxetex
    \usepackage{mathspec}
  \else
    \usepackage{fontspec}
  \fi
  \defaultfontfeatures{Ligatures=TeX,Scale=MatchLowercase}
\fi
% use upquote if available, for straight quotes in verbatim environments
\IfFileExists{upquote.sty}{\usepackage{upquote}}{}
% use microtype if available
\IfFileExists{microtype.sty}{%
\usepackage{microtype}
\UseMicrotypeSet[protrusion]{basicmath} % disable protrusion for tt fonts
}{}
\usepackage[left=3.5cm,right=2.5cm,top=2.5cm,bottom=2.5cm]{geometry}
\usepackage{hyperref}
\PassOptionsToPackage{usenames,dvipsnames}{color} % color is loaded by hyperref
\hypersetup{unicode=true,
            pdftitle={A Questão Agrária},
            pdfauthor={Luiz F. P. Droubi},
            colorlinks=true,
            linkcolor=red,
            citecolor=green,
            urlcolor=magenta,
            breaklinks=true}
\urlstyle{same}  % don't use monospace font for urls
\usepackage{longtable,booktabs}
\usepackage{graphicx,grffile}
\makeatletter
\def\maxwidth{\ifdim\Gin@nat@width>\linewidth\linewidth\else\Gin@nat@width\fi}
\def\maxheight{\ifdim\Gin@nat@height>\textheight\textheight\else\Gin@nat@height\fi}
\makeatother
% Scale images if necessary, so that they will not overflow the page
% margins by default, and it is still possible to overwrite the defaults
% using explicit options in \includegraphics[width, height, ...]{}
\setkeys{Gin}{width=\maxwidth,height=\maxheight,keepaspectratio}
\IfFileExists{parskip.sty}{%
\usepackage{parskip}
}{% else
\setlength{\parindent}{0pt}
\setlength{\parskip}{6pt plus 2pt minus 1pt}
}
\setlength{\emergencystretch}{3em}  % prevent overfull lines
\providecommand{\tightlist}{%
  \setlength{\itemsep}{0pt}\setlength{\parskip}{0pt}}
\setcounter{secnumdepth}{5}
% Redefines (sub)paragraphs to behave more like sections
\ifx\paragraph\undefined\else
\let\oldparagraph\paragraph
\renewcommand{\paragraph}[1]{\oldparagraph{#1}\mbox{}}
\fi
\ifx\subparagraph\undefined\else
\let\oldsubparagraph\subparagraph
\renewcommand{\subparagraph}[1]{\oldsubparagraph{#1}\mbox{}}
\fi

%%% Use protect on footnotes to avoid problems with footnotes in titles
\let\rmarkdownfootnote\footnote%
\def\footnote{\protect\rmarkdownfootnote}

%%% Change title format to be more compact
\usepackage{titling}

% Create subtitle command for use in maketitle
\providecommand{\subtitle}[1]{
  \posttitle{
    \begin{center}\large#1\end{center}
    }
}

\setlength{\droptitle}{-2em}

  \title{A Questão Agrária}
    \pretitle{\vspace{\droptitle}\centering\huge}
  \posttitle{\par}
  \subtitle{no Brasil e no Mundo}
  \author{Luiz F. P. Droubi}
    \preauthor{\centering\large\emph}
  \postauthor{\par}
      \predate{\centering\large\emph}
  \postdate{\par}
    \date{13/05/2019}

\usepackage[brazil]{babel}
\usepackage{graphicx}
\usepackage{float}
\usepackage{subfig}
\usepackage{caption}
\usepackage{lastpage}
\setlength{\parindent}{1.25cm} % Default is 15pt.
\usepackage{indentfirst}
\usepackage{mathptmx} % para Times New Roman
%\usepackage{fontspec} % para Arial
%\setmainfont{Arial}
\newcommand{\pkg}[1]{{\normalfont\fontseries{b}\selectfont #1}}
\let\proglang=\textsf
\let\code=\texttt
\usepackage{fancyhdr}
% Turn on the style
\pagestyle{fancy}
% Clear the header and footer
\fancyhead{}
\fancyfoot{}
% Set the right side of the footer to be the page number
\fancyfoot[R]{\thepage~/~\pageref{LastPage}}

\begin{document}
\maketitle

\hypertarget{resumo}{%
\section*{Resumo}\label{resumo}}
\addcontentsline{toc}{section}{Resumo}

Este trabalho foi desenvolvido para obtenção de conceito na Disciplina
TGT4100181 do Programa de Pós-Graduação em Engenharia de Transportes e
Gestão Territorial da UFSC -- Cadastro Técnico Multifinalitário. Teve
como motivação as três condições, elencadas por Pereira
(\protect\hyperlink{ref-questaoagraria}{1987}), como necessárias à
implantação da reforma agrária, quais sejam: \textbf{a.} que haja no
país uma alta concentração de renda; \textbf{b.} que haja verdadeira
miséria rural no País e; \textbf{c.} que haja relação entre a reforma
agrária e o desenvolvimento nacional. Pra nós claro está que as
condições \textbf{a.} e \textbf{b.} estão desde há muito tempo intocadas
no Brasil, ou seja, estão presentes desde a origem. Coube-nos investigar
se há verdadeira relação, ainda, entre a reforma e o desenvolvimento.

\hypertarget{introducao}{%
\section{Introdução}\label{introducao}}

\begin{quote}
Do ponto de vista social, todos os fatores se resumem em um ``recurso''
elementar: o homem. Logo, não é possível seguir conceptualmente o
processo de industrialização se não sabemos como o homem aplicava antes
o seu tempo de trabalho, como o aplica depois, o que ocorre quando passa
de um modo de produzir a outra e em que condições realiza essa
passagem.{[}\ldots{}{]} Considerando que na estrutura da economia que
precede a industrialização quase toda a população está na
``agricultura'', é preciso estudar detidamente a organização deste
setor. Em outras palavras, se o problema da ``agricultura'' não foi
entendido, tampouco será possível compreender o problema da
``indústria'', ou manufatura, nem o papel que os serviços desempenham.
Falando de modo sucinto, a ``manufatura'' e os serviços são novas formas
de aplicação de parte do tempo de trabalho da população que antes estava
na ``agricultura''. Mas, por sua vez, a própria ``agricultura'' se
reorganiza quando a tranferência ocorre.(RANGEL,
\protect\hyperlink{ref-rangel1954}{2012}\protect\hyperlink{ref-rangel1954}{a},
p. 89)
\end{quote}

O desenvolvimento do capitalismo brasileiro no século XX se deu pela
chamada ``via prussiana'' ou \emph{junker} (RANGEL,
\protect\hyperlink{ref-rangel1988}{2012}\protect\hyperlink{ref-rangel1988}{b},
p. 155), que é um tipo de reforma agrária que consiste na substituição
do latifúndio feudal pelo latifúndio capitalista. Este tipo de
desenvolvimento tem como característica se dar sem a execução prévia da
reforma agrária -- no sentido da distribuição dos latifúndios em
pequenas propriedades, como ocorre na via clássica ou democrática.

Apesar de permitir imprimir um ``impulso extraordinário e energético'' à
industrialização, a via prussiana ``promove uma distribuição muito
desigual da renda'' (RANGEL,
\protect\hyperlink{ref-rangel1988}{2012}\protect\hyperlink{ref-rangel1988}{b},
p. 155). característica talvez mais perniciosa do desenvolvimento
capitalista por esta via se dá pela formação de um ``exército industrial
de reserva'' demasiado grande, ou seja, um aumento da população urbana
desproporcional à necessidade de mão-de-obra necessária nas indústrias
do capitalismo nascente nas cidades. O resultado é o crescimento
acelerado e desordenado das cidades, com a inevitável formação dos
cortiços e favelas para acomodar a parte mais carente da população que,
expulsa do campo, vai se aglomerar nos grandes centros urbanos em busca
da sua sobrevivência.

Dados compilados pela ONU foram organizados na tabela abaixo com o
intuito de demonstrar o tamanho exato deste problema.

\begin{longtable}[]{@{}lcccccc@{}}
\caption{População urbana (\%). Fonte: O autor.}\tabularnewline
\toprule
País/Região & Ano & & & & &\tabularnewline
\midrule
\endfirsthead
\toprule
País/Região & Ano & & & & &\tabularnewline
\midrule
\endhead
& 1960 & 1970 & 1980 & 1990 & 2000 & 2015\tabularnewline
Mundo & 33,8\% & 36,6\% & 39,3\% & 43,0\% & 46,7\% &
53,9\%\tabularnewline
Países desenvolvidos (a) & 61,1\% & 66,8\% & 70,3\% & 72,4\% & 74,2\% &
78,1\%\tabularnewline
Países menos desenvolvidos (b) & 21,9\% & 25,3\% & 29,4\% & 34,9\% &
40,1\% & 49,0\%\tabularnewline
Europa & 57,4\% & 63,1\% & 67,6\% & 69,9\% & 71,1\% &
73,9\%\tabularnewline
Europa oriental & 48,9\% & 56,6\% & 63,8\% & 68,0\% & 68,2\% &
69,3\%\tabularnewline
Europa ocidental & 68,6\% & 72,1\% & 73,4\% & 74,0\% & 76,0\% &
79,4\%\tabularnewline
EUA & 70,0\% & 73,6\% & 73,7\% & 75,3\% & 79,1\% & 81,7\%\tabularnewline
Brasil & 46,1\% & 55,9\% & 65,5\% & 73,9\% & 81,2\% &
85,8\%\tabularnewline
\bottomrule
\end{longtable}

\begin{enumerate}
\def\labelenumi{\alph{enumi})}
\tightlist
\item
  Europa, América do Norte, Austrália/Nova Zelândia e Japão.\\
\item
  Africa, Ásia (menos Japão), América Latina e Caribe mais Melanesia e
  Micronesia.
\end{enumerate}

Em meados dos anos 60, apenas 46,1\% da população brasileira era urbana,
uma proporção bem menor do que a dos países do então primeiro mundo (EUA
e Europa Ocidental), hoje ditos desenvolvidos, que girava então em torno
dos 70\% da população.

Em apenas 10 anos, já em meados da década de 70, este número sofria um
aumento vertiginoso de quase 10 pontos percentuais, com 55,9\% da
população urbana. A população urbana brasileira equiparava-se à da
Europa Oriental.

Já na década de 80 a população urbana no Brasil ultrapassaria a da
Europa Oriental, chegando à valores próximos da média para o continente
europeu como um todo (ocidental e oriental), enquanto a população urbana
no mundo desenvolvido se estagnava.

Chegado os anos 90, a população urbana brasileira atingiu notáveis
73,9\% da população brasileira, número equiparado ao da população urbana
do mundo desenvolvido (74\% na Europa Oriental).

Em meados dos anos 2000, já então no século atual, ousamos ultrapassar,
em proporção, a população urbana da Europa Oriental e a dos EUA,
chegando ao último dado de 2015, com 85,8\% da população brasileira
vivendo nas cidades.

Há de se levar em consideração, ainda, que este ``êxodo rural'' ainda
foi acompanhado de um crescimento demográfico expressivo.

Todo este crescimento expressivo seria salutar se tivesse se dado no
contexto do rápido desenvolvimento da economia nacional. Isto, porém,
não ocorreu durante todo o período analisado. O crescimento da economia
brasileira acelerou-se na segunda quadra da década de 60 e manteve-se
alto até fins da década seguinte, porém estagnou-se na década de 80, a
chamada década perdida, sem que com isso a população urbana deixasse de
crescer vertiginosamente.

Para Rangel (RANGEL,
\protect\hyperlink{ref-rangel1986a}{2012}\protect\hyperlink{ref-rangel1986a}{c},
p. 151), ``essa redistribuição da população entre os quadros urbano e
rural não tem, em si mesma, nada de anormal.{[}\ldots{}{]} A
urbanização, em si mesma, é um fenômeno perfeitamente normal, numa
economia em processo de industrialização. O que não é normal é o ritmo
que imprimimos ao \emph{nosso} processo de urbanização, que implica
criar, nas cidades, uma oferta de mão-de-obra em descompasso com a
demanda que a industrialização vai criando.''

Todo este processo só poderia, então, ter desaguado no inchaço das
principais cidades brasileiras. Desnecessário dizer que o planejamento
urbano nestas condições é praticamente inviável. As administrações
municipais, nem que fossem as mais eficientes, teriam capacidade de
planejar e disciplinar o uso do solo urbano nesta ``velocidade
migratória''.

Para Rangel
(\protect\hyperlink{ref-rangel1988}{2012}\protect\hyperlink{ref-rangel1988}{b},
pp. 156--157), com o desenvolvimento da indústria pesada no Brasil, a
crise agrária, antes cíclica, tornou-se crônica, criando um ``vasto
deslocamento de população, na direção geral campo-cidade. Esse movimento
se faz escalonadamente, das áreas rurais para as cidades pequenas;
destas para as médias e grandes, e posteriormente para as metrópoles
gigantes. No fim da linha, portanto, vamos encontrar as cidades de São
Paulo e do Rio de Janeiro''. Enfim, para Rangel, a origem deste
``multidinário deslocamento demográfico'', está ``o modo como o país
preparou sua estrutura agrária para a industrialização''.

\hypertarget{contexto-historico}{%
\section{Contexto histórico}\label{contexto-historico}}

O Capitalismo é um sistema político-econômico que, historicamente,
substitui o Feudalismo, sistema em que a população encontrava-se toda
concentrada no campo.

Nas sociedades pré-capitalistas, a população predominante rural
organizava-se no chamado `Complexo Rural', ou seja, eram produzidos pelo
camponês não apenas os produtos agrícolas, mas todo o ferramental
necessário, suas vestes e utensílios.

A passagem do sistema feudal para o sistema capitalista ocorre com a
divisão social do trabalho, ou seja, com o desenvolvimento de indústrias
que vão aos poucos absorver as atividades não-agrícolas realizadas no
campo.

\begin{quote}
Numa economia em expansão, com crescente industrialização,
comercialização e urbanização, numerosos processos anteriormente levados
a efeito antes dentro da casa da família ou unidade (econômica)
famíliar, ou são completamente abandonados ou substituídos por processos
semelhantes em bases comerciais.(KUZNETS
(\protect\hyperlink{ref-kuznets}{1952}, p. 41) \emph{apud} RANGEL
(\protect\hyperlink{ref-rangel1956}{2012}\protect\hyperlink{ref-rangel1956}{d},
p. 218))
\end{quote}

\hypertarget{feudalismo}{%
\section{Feudalismo}\label{feudalismo}}

No feudalismo, tipo de organização social que historicamente antecede o
capitalismo, toda terra está concentrada nas mãos do rei.

As ``leis'' que regem uma sociedade feudal são:

\begin{itemize}
\tightlist
\item
  \emph{All land is king's land}
\item
  \emph{Nulle terre sans seigneur}
\end{itemize}

Segundo Rangel
(\protect\hyperlink{ref-rangel1961}{2012}\protect\hyperlink{ref-rangel1961}{e},
p. 219), ``a existência de terra livre é incompatível com o
feudalismo'', e , por conta disto, ``nas condições feudais, não tem
preço e é, de fato ou de direito, inalienável''(RANGEL,
\protect\hyperlink{ref-rangel1960}{2012}\protect\hyperlink{ref-rangel1960}{f},
p. 206).

\hypertarget{a-crise-do-feudalismo}{%
\subsection{A crise do feudalismo}\label{a-crise-do-feudalismo}}

A sociedade feudal entra em crise quando a produção agrícola não
consegue suprir a demanda da superpopulação gerada.

Segundo Rangel
(\protect\hyperlink{ref-rangel1961}{2012}\protect\hyperlink{ref-rangel1961}{e},
p. 219), ``tempo houve em que a expansão do estoque populacional era
objetivamente a maneira mais eficaz de expandir as forças produtivas e o
produto social. Nesse tempo (regime feudal), a riqueza dos príncipes se
media pelas almas dos seus domínios, e aumentar o número destas era a
maneira óbvia de expandir aquela riqueza e também a do corpo social.
Este foi forjando para si uma ética, um direito e uma política
conducentes a esse resutado''.

\hypertarget{o-complexo-rural}{%
\subsection{O Complexo Rural}\label{o-complexo-rural}}

\begin{quote}
A população de um páis de economia mercantil debilmente desenvolvida (ou
não densevolvida de todo) é quase exclusivamente agrícola. Todavia, não
se deve deduzir daí que ela se ocupa só da agricultura. Significa apenas
que a população ocupada na agricultura transforma, ela mesma, os
produtos da terra, sendo quase inexistentes o intercâmbio e a divisão do
trabalho. (LENIN \emph{apud} RANGEL
(\protect\hyperlink{ref-rangel1954}{2012}\protect\hyperlink{ref-rangel1954}{a},
p. 99))
\end{quote}

Segundo Rangel
(\protect\hyperlink{ref-rangel1956}{2012}\protect\hyperlink{ref-rangel1956}{d},
p. 98), a unidade agrícola fechada é ``um microcosmo econômico no qual
as pessoas distribuem seu tempo entre numerosas atividades. Cada uma
dessas atividades representa o estado rudimentar daquilo que, com o
desenvolvimento, se tornará uma `indústria' (\ldots{}) É evidente que o
camponês não tem consciência da multiplicidade de suas atividades. Ele
considera que elas formam um todo indivisível. Essa inespecialização é
sua especialidade. (\ldots{}) Chamaremos esse microcosmo econômico, essa
`matriz de insumo-produto' em miniatura, de `complexo rural'\,''.

\hypertarget{condicoes-e-metodos-de-abertura-do-complexo-rural}{%
\subsection{Condições e Métodos de abertura do Complexo
Rural}\label{condicoes-e-metodos-de-abertura-do-complexo-rural}}

\begin{quote}
A Abertura do Complexo Rural não é uma operação momentânea, mas sim um
largo processo, com altos e baixos e problemas sempre novos. Sua
história está muito longe de ser idílica. Ao contrário, está cheia de
violência. Uma planificação econômica que não resolva preliminarmente
este problema é inconcebível. Alternadamente, pode conduzir à liberação
de mais fatores que aqueles que os setores não agrícolas podem usar,
fazendo toda a economia submergir em uma crise profunda, ou condenar
esses setores à estagnação por insuficiência de fatores.(RANGEL,
\protect\hyperlink{ref-rangel1954}{2012}\protect\hyperlink{ref-rangel1954}{a},
p. 118)
\end{quote}

Para a abertura do Complexo Rural é necessário que haja vantajosidade
para a economia de mercado e para a economia natural do próprio
Complexo.

As medidas tendentes a romper o complexo rural podem ser classificadas
em dois grupos (RANGEL,
\protect\hyperlink{ref-rangel1954}{2012}\protect\hyperlink{ref-rangel1954}{a},
p. 113):

\begin{enumerate}
\def\labelenumi{\alph{enumi}.}
\item
  as que oferecem um incentivo positivo para a incorporação, à economia
  de mercado, dos fatores usados pelo complexo e;
\item
  as que buscam forçar a abertura do complexo a partir de dentro,
  provocando uma deterioração da produtividade das atividades
  manufatureiras dentro do complexo.
\end{enumerate}

As medidas do tipo a) temm seu exemplo mais típico nos EUA e também na
França, enquanto as medidas do tipo b) predominaram na Inglaterra,
Alemanha e Japão (RANGEL,
\protect\hyperlink{ref-rangel1954}{2012}\protect\hyperlink{ref-rangel1954}{a},
pp. 114--115).

\hypertarget{exodo-rural-e-industrializacao}{%
\section{Êxodo rural e
industrialização}\label{exodo-rural-e-industrializacao}}

\begin{quote}
A revolução democrático-burguesa, nos casos em que a gleba feudal é --
como aconteceu na Europa Ocidental (principalmente, na França) e nos
Estados Unidos -- substituída pela pequena propriedade familiar ou
\emph{homestead}, ao fortalecer as bases da economia natural ou de
autoconsumo, resolve satisfatoriamente o problema na absorção dos
excedentes de mão-de-obra no seio da própria economia camponesa,
estancando ou reduzindo drasticamente o fluxo populacional responsálve
pelo êxodo campo-cidade (RANGEL,
\protect\hyperlink{ref-rangel1986b}{2012}\protect\hyperlink{ref-rangel1986b}{g},
p. 133).
\end{quote}

Segundo, Rangel, no entanto, ``esse tipo de superação das relações de
produção feudais não é característico do Brasil. Sem embargo do
surgimento de algumas `ilhas' de pequena propriedade camponse,
notadamente nas áreas de colonização européia e japonesa nos estados do
Sul, que mais confirmam a regra.''

Ocorre que, de acordo com Rangel, ``a superabundância e a barateza da
mão-de-obra não costumam ser bons condicionanetes do processo de
industrialização, dado que desestimulam a formação de capital, isto é, o
inestimento. Ora, numa economia capitalista, o investimento é o motor
primário do desenvolvimento \ldots{}''.

Por este motivo, a ``economia brasileira, nas condições de uma crise
agrária profunda e crônica que, entre outras coisas, \textbf{causava uma
urbanização monstruosa}, sem comparação possível com a demanda de
mão-de-obra que a indústria e os serviços não-agrícolas estavam
suscitando nas cidades (perto de três milhões de novos citadinos a cada
ano)\ldots{}''

\hypertarget{o-exodo-rural-como-obstaculo-ao-desenvolvimento}{%
\subsection{O êxodo rural como obstáculo ao
desenvolvimento}\label{o-exodo-rural-como-obstaculo-ao-desenvolvimento}}

No capitalismo, é conhecido o papel do investimento ou formação de
capital nas taxas de desemprego. Segundo Rangel
(\protect\hyperlink{ref-rangel1988}{2012}\protect\hyperlink{ref-rangel1988}{b},
p. 156), ``por um lado, via efeito multiplicador (efeito para trás), o
investimento cria emprego de mão-de-obra; por outro lado, via
implementação de nova tecnologia, promove dispensa de mão-de-obra
(efeito para frente)''.

Segundo Rangel (RANGEL,
\protect\hyperlink{ref-rangel1986c}{2012}\protect\hyperlink{ref-rangel1986c}{h},
p. 142), um ``\,`exército industrial de reserva' limitado, isto é, algum
desemprego, pode ser considerado útil, do ponto de vista da produção
capitalista, porque serve de instrumento de coerção para os
trabalhadores livres, fortalecendo assim a disciplina no trabalho''. No
entanto, quando este torna-se excessivo, ``pode converter-se em
obstáculo ao desenvolvimento da própria economia capitalista. Ora, aqui
está o nosso problema, dado que o `exército industrial de reserva'
brasileiro tornou-se teratologicamente grande. Por isso mesmo, a questão
agrária, que se exprime precipuamente pela formação desse `exército',
não interessa apenas aos camponeses, mas à sociedade como um todo.''

De acordo com Rangel
(\protect\hyperlink{ref-rangel1988}{2012}\protect\hyperlink{ref-rangel1988}{b},
p. 156), ``a via democrática -- divisão dos latifúndios em pequenas
propriedades -- ao favorecer uma distribuição menos desigualitária de re
da cria condições para um vigoroso efeito multiplicador dos
investimentos, isto é, forte efeito para trás. Inversamente, a via
prussiana, ao promover uma distribuição de renda mais desigualitária,
debilita o efeito multiplicador, isto é, para trás, mas, por força da
concentração de renda, aumenta o peso relativo dos investimentos
dispensando mão-de-obra e, por isso mesmo, aumentando o efeito para
diante.''

\hypertarget{referencias}{%
\section*{Referências}\label{referencias}}
\addcontentsline{toc}{section}{Referências}

\hypertarget{refs}{}
\leavevmode\hypertarget{ref-kuznets}{}%
KUZNETS, S. \textbf{Long term changes in the national income of the
united state of america, since 1870}. Income \& Wealth, 1952.

\leavevmode\hypertarget{ref-questaoagraria}{}%
PEREIRA, F. J. Algumas questões em torno da reforma agrária. In:
\textbf{A Questão Agrária e o Desenvolvimento Nacional}. p.11--15, 1987.
Florianópolis: SUDESUL; Editora da UFSC.

\leavevmode\hypertarget{ref-rangel1954}{}%
RANGEL, I. O desenvolvimento econômico no Brasil. In: \textbf{Ignácio
rangel: Obras reunidas}. 3rd ed., v. 1, p.39--128, 2012a. Rio de
Janeiro: César Benjamin; Contraponto.

\leavevmode\hypertarget{ref-rangel1988}{}%
RANGEL, I. Fim de linha. In: \textbf{Ignácio rangel: Obras reunidas}.
3rd ed., v. 2, p.155--157, 2012b. Rio de Janeiro: César Benjamin;
Contraponto.

\leavevmode\hypertarget{ref-rangel1986a}{}%
RANGEL, I. Crise agrária e metrópole. In: \textbf{Ignácio rangel: Obras
reunidas}. 3rd ed., v. 2, p.149--155, 2012c. Rio de Janeiro: César
Benjamin; Contraponto.

\leavevmode\hypertarget{ref-rangel1956}{}%
RANGEL, I. Desenvolvimento e Projeto. In: \textbf{Ignácio rangel: Obras
reunidas}. 3rd ed., v. 1, p.203--283, 2012d. Rio de Janeiro: César
Benjamin; Contraponto.

\leavevmode\hypertarget{ref-rangel1961}{}%
RANGEL, I. Demografia e democracia. In: \textbf{Ignácio rangel: Obras
reunidas}. 3rd ed., v. 2, p.218--220, 2012e. Rio de Janeiro: César
Benjamin; Contraponto.

\leavevmode\hypertarget{ref-rangel1960}{}%
RANGEL, I. Depoimento sobre a Questão Agrária. In: \textbf{Ignácio
rangel: Obras reunidas}. 3rd ed., v. 2, p.205--207, 2012f. Rio de
Janeiro: César Benjamin; Contraponto.

\leavevmode\hypertarget{ref-rangel1986b}{}%
RANGEL, I. A Questão Agrária e o ciclo longo. In: \textbf{Ignácio
rangel: Obras reunidas}. 3rd ed., v. 2, p.129--140, 2012g. Rio de
Janeiro: César Benjamin; Contraponto.

\leavevmode\hypertarget{ref-rangel1986c}{}%
RANGEL, I. A Questão da Terra. In: \textbf{Ignácio rangel: Obras
reunidas}. 3rd ed., v. 2, p.141--149, 2012h. Rio de Janeiro: César
Benjamin; Contraponto.


\end{document}
